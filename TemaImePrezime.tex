% !TEX encoding = UTF-8 Unicode

\documentclass[a4paper]{article}

\usepackage{color}
\usepackage{url}
%\usepackage[T2A]{fontenc} % enable Cyrillic fonts
\usepackage[utf8]{inputenc} % make weird characters work
\usepackage{graphicx}

\usepackage[english,serbian]{babel}
%\usepackage[english,serbianc]{babel} %ukljuciti babel sa ovim opcijama, umesto gornjim, ukoliko se koristi cirilica

\usepackage[unicode]{hyperref}
\hypersetup{colorlinks,citecolor=green,filecolor=green,linkcolor=blue,urlcolor=blue}

%\newtheorem{primer}{Пример}[section] %ćirilični primer
\newtheorem{primer}{Primer}[section]

\begin{document}

\title{Naslov seminarskog rada\\ \small{Seminarski rad u okviru kursa\\Metodologija stručnog i naučnog rada\\ Matematički fakultet}}

\author{Milomir Radojević, Marija Živković\\ milomir993@gmail.com, marijaz93@yahoo.com}
\date{12.~april 2015.}
\maketitle

\abstract{
Android je danas najpopularniji operativni sistem za pametne telefone i tablet računare(X). Takođe se koristi u televiziji, automobilima, digitalnim kamerama i mnogim drugim uređajima. Osim aplikacija koje služe da korisnicima olakšaju poslovanje, organizaciju obaveza ili komunikaciju, postoji veliki broj aplikacija čija je namena zabava. Aplikacije za manipulaciju multimedijalnim sadržajima i igre su samo neke od njih. Ovaj rad opisuje neke principe razvoja igara na Android platformi, daje smernice za određivanje osnovnih elemenata arhitekture igre, i prati primer implementacije konktretne igre "Battleships". Takođe se demonstrira višekorisnički režim rada aplikacije razmenom podataka preko Bluetooth-a.

\tableofcontents

\newpage

\section{Uvod}
\label{sec:uvod}
Tipična arhitektura bilo koje igre izgleda ovako(X):
\begin{enumerate}
\item prihvatanje unosa od strane korisnika;
\item ažuriranje stanja objekata (računanje nove pozicije objekta, provera logičkih uslova, ispitivanje kolizije i slično);
\item emitovanje zvuka;
\item iscrtavanje slike na ekran;
\item ponavljanje prethodnih koraka dok traje igra.
Jedan od osnovnih zadataka u implementaciji je izgradnja glavne petlje koja će izvršavati prethodno navedene korake. U Android-u, sve kreće od klase Activity(X). Activity uglavnom služi za interakciju sa korisnikom. Iz tog razloga Activity se stara za pravljenje prozora u koji se smešta korisnički interfejs (View). View(X) je mesto gde se prihvata unos od strane korisnika i gde grafički elementi bivaju prikazani. Prethodni opis je prikazan na dijagramu(X). Prozor (koji u stvari predstavlja klasu koja nasleđuje View), kao atribut sadrži instancu klase (koja nasleđuje klasu Thread) koja periodično poziva metode update() i draw() koji ažuriraju stanje i prikazuju elemente prozora. Prihvatanje unosa se vrši pomoću metoda onTouchEvent() dok se iscrtavanje na ekran vrši korišćenjem klase Canvas. Višekorisnički režim se ostvaruje pomoću klasa iz paketa android.bluetooth.
\end{enumerate}


\section{Slike i tabele}
\label{slike_i_tabele}

Slike i tabele treba da budu u svom okruženju, sa odgovarajućim naslovima, obeležene labelom da koje omogućava referenciranje. 

\begin{primer} Ovako se ubacuje slika. Obratiti pažnju da je dodato i 
\begin{verbatim}
\usepackage{graphicx}
\end{verbatim}

\begin{figure}[h!]
\begin{center}
%\includegraphics[scale=0.75]{panda.jpg}
\end{center}
\caption{Pande}
\label{fig:pande}
\end{figure}

Na svaku sliku neophodno je referisati se negde u tekstu. Na primer, na slici \ref{fig:pande} prikazane su pande. 
\end{primer}

\begin{primer} I tabele treba da budu u svom okruženju, i na njih je neophodno referisati se u tekstu. Na primer, u tabeli \ref{tab:tabela1} su prikazana različita poravnanja u tabelama.

\begin{table}[h!]
\begin{center}
\caption{Razlčita poravnanja u okviru iste tabele ne treba koristiti jer su nepregledna.}
\begin{tabular}{|c|l|r|} \hline
centralno poravnanje& levo poravnanje& desno poravnanje\\ \hline
a &b&c\\ \hline
d &e&f\\ \hline
\end{tabular}
\label{tab:tabela1}
\end{center}
\end{table}

\end{primer}





\section{Glavna petlja(X)}
\label{sec:naslov1}


Za prozor u Activity klasi stvaramo klasu GameSurface koja nasleđuje SurfaceView (i implementira SurfaceHolder.Callback da mogla da reaguje na događaje kao što su promena orijentacije uređaja ili pravljenje i uništavanje GameSurface objekta). Pri pravljenu GameSurface objekta, pokreće se klasa koja sadrži glavnu petlju, dok se pri uništenju ona prekida. U toj klasi se nalazi beskonačna petlja u kojoj se redom u određenom vremenskom intervalu pozivaju update() i draw() metodi GameSurface objekta. GameSurface objekat poziva odgovarajuće update() i draw() metode za sve svoje elemente.


\subsection{Prihvatanje unosa od korisnika(X)}
\label{subsec:podnaslov1}

Prihvatanje unosa se vrši metodom onTouchEvent tako što se za svaki događaj prosledi informacija odgovarajućem objektu koji će da ga obradi.

\subsection{Iscrtavanje slike na ekran(X)}
\label{subsec:podnaslov2}

Iscrtavanje se vrši pomoću metoda draw() i objekta Canvas koji omogućuje prikazivanje slika i menjanje piksela na ekranu. Kada se pozove metod draw() GameSurface objekta i prosledi mu se Canvas, Canvas se zaključa i GameSurface objekat može da crta po njemu. Zaključavanjem se obezbeđuje da je Canvas nepromenljiv dok se po njemu crta.

\section{Ostvarivanje višekorisničkog režima preko Bluetooth-a(X)}
\label{sec:naslov2}

Ovde pišem tekst. 
Ovde pišem tekst. 
Ovde pišem tekst. 
Ovde pišem tekst. 

\subsection{... podnaslov}
\label{subsec:podnaslovN}

Ovde pišem tekst. 
Ovde pišem tekst. 
Ovde pišem tekst. 
Ovde pišem tekst. 
Ovde pišem tekst. 
Ovde pišem tekst. 

\section{n-ti naslov}
\label{sec:naslovN}

Ovde pišem tekst. 
Ovde pišem tekst. 
Ovde pišem tekst. 
Ovde pišem tekst. 
Ovde pišem tekst. 

\subsection{... podnaslov}
\label{subsec:podnaslovK}

Ovde pišem tekst. 
Ovde pišem tekst. 
Ovde pišem tekst. 
Ovde pišem tekst. 
Ovde pišem tekst. 

\subsection{... podnaslov}
\label{subsec:podnaslovM}

Ovde pišem tekst. 
Ovde pišem tekst. 
Ovde pišem tekst. 
Ovde pišem tekst. 
Ovde pišem tekst. 

\section{Poslednji naslov}
\label{sec:naslovM}

Ovde pišem tekst. 
Ovde pišem tekst. 
Ovde pišem tekst. 
Ovde pišem tekst. 
Ovde pišem tekst. 
Ovde pišem tekst. 
Ovde pišem tekst. 
Ovde pišem tekst. 
Ovde pišem tekst. 

\section{Zaključak}
\label{sec:zakljucak}

Prikazana je jednostavna arhitektura koja može biti upotrebljena za razvoj velikog broja Android igara i sličnih aplikacija. Naveden je jedan način na koji je moguće realizovati glavnu petlju, prihvatanje korisničkog unosa, prikazivanje grafičkih elemenata na ekran i povezivanje dva uređaja preko mreže. Opisani su elementi koji predstavljaju temelj na kom se može zasnovati širok spektar različitih aplikacija. Višestruka upotrebljivost se postiže nadogradnjom nekih od prethodno opisanih koncepata i dodavanjem elemenata koji unapređuju mogućnosti animacije, simulacije fizičkih zakona i drugih poboljšanja. Ovaj rad je pokušaj da se čitaocima koji poznaju programski jezik Java a koji se nisu ranije susretali sa razvojem aplikacija za Android platformu, približi proces implementacije jednostavne igre i pokažu neke osnovne karakteristike programiranja za Android.


\addcontentsline{toc}{section}{Literatura}
\appendix
\bibliography{seminarski} 
\bibliographystyle{plain}

\appendix
\section{Dodatak}
Ovde pišem dodatne stvari, ukoliko za time ima potrebe.
Ovde pišem dodatne stvari, ukoliko za time ima potrebe.
Ovde pišem dodatne stvari, ukoliko za time ima potrebe.
Ovde pišem dodatne stvari, ukoliko za time ima potrebe.
Ovde pišem dodatne stvari, ukoliko za time ima potrebe.


\end{document}
